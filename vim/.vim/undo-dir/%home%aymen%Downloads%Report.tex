Vim�UnDo���&���T򥪪��~��P}B�DS��~�^!?_�P����PPV^!>�Oj�This study addresses the problem of global warming and climate change. The use of microalgae for CO$_2$ utilisation from carbon capture (CCU) presents a promising method of mitigating CO$_2$ emissions. As microalgae have applications in a wide range of industries, they allow value to be added to high cost carbon capture processes. The biochemical processes involved in the growth mechanisms of microalgae are complex, and so developing efficient cultivation systems for CO$_2$ utilisation can be difficult. Modelling microalgae growth allows the process to be simulated to determine how it can be optimised. A mathematical model was, therefore, developed to model the growth of microalgae. The model expressions were developed to simulate the algae cultivation in a continuous bioreactor system. Once an initial solution to the model had been found, the model was validated with experimental data for cultivation of the microalgae \textit{Chlorella}.  The experiments had measured microalgae cultivation with varying feed gas concentrations: air, 5\% CO$_2$, 10\% CO$_2$ and 15\% CO$_2$. Seven parameters which were deemed to be most important for microalgae growth were chosen, and a parameter estimation was then performed to fit the model to the experimental data. The resulting optimum parameter values were found to be similar to those in previous studies concerning the same microalgae species. It was found that the 5\% CO$_2 $ fed mircoalgae produced the highest growth rates and biomass concentrations. This study also shows that increasing the CO$_2$ concentration above this value reduces biomass production. A sensitivity analysis was also performed and showed that biomass concentration was most sensitive to the specific growth rate parameters. It also showed that the light intensity parameters did not affect biomass production as much as initially expected. Further light based experiments would, therefore, give insight into how light intensity can be optimised to maximise microalgae growth.5��